\chapter{The Drone Incident}

\noindent Hey Dad,

I've always thought there are two types of people: those with too many stories but not enough time to tell them all, and those with too much time, but too few stories to tell.
I suppose there is also a third group of people who talk too much, but have nothing actually interesting to say.
And there is probably a fourth group who have very interesting stories, but never bother to tell anyone about them.

Well, I really got lost in my own analogy there.
My point is that there are people who live their lives attempting to do the most interesting thing they can imagine, and those who don't.
I always wanted to be a person who did things, but can't honestly say that's true.
I feel like if anyone was a don'ter, it was me\footnote{Geez, what is this writing?}

You, on the other hand, were a doer.
I still have your collection of absolutely ancient technology sitting in the cellar.
In fact, I even have some of the more interesting pieces displayed in the living room in the case of company.
Your flip phone is a personal favorite of mine, the one with the Guiness record for the ``World's oldest functional phone.''
It feels more like a toy than a usable device, but it still technically ``works,'' whatever that means.
I mean, you modded it to have a useable touch-panel with a decent connection to the internet, but it's still too slow to do anything meaningful.
It was also made in the era before phones were quantum compatable, so it's certainly not secure enough to hold any ID info.
I guess because of that, you didn't put in a credit chip, so it can't pay for anything either.
It doesn't even have the processing power to connect to a projector!
The only thing it can do is call the police in an emergency, but I guess that's the point.
I don't know what you can meaningfully expect a phone from 2008 to do.

Anyway, it's sitting on the shelf, right next to the certificate.
I simply cannot imagine an era where people actually bought hardware in the first place!
It's one of the many things I'm glad we were able to streamline in the past century or so.
It's a relic, that's for sure.

It was one of the many things you would tinker about with in the workshop.
You had everything there: 3D printers, breadboards, a laser-cutting machine...
You always said it was ``better to make something yourself than to rely on someone else to make it for you.
Even if the thing you make is worse than something you could buy off the shelf, it's truly *yours*.''
You would then go on a rant about how the world became worse and worse after everyone lost the right to own their devices and how you would forever refuse to use any technology without it's schematics in-hand.

Can I just pause for a second?

I understand you had rules.
I understand wanting to enforce those rules within your family.
But some of those rules were out-of-line.

Like, why was \textit{I} never allowed to purchase games just because \textit{you} didn't understand the software?
That \textit{sucked} as a kid.
No matter how hard I tried, I could never convince you to buy \textit{anything} at the store.
Games, AI, projectors... You flat-out refused to buy them.
I had to save up my own money for just about anything I wanted!

Okay, to be honest, it was probably better that way.
It motivated me to get some summer jobs during high school and keep track of my own finances.
I guess I would have lost out on all that had you bent the rules.

Yeah, I guess it was fine, even if I hated it as a kid.

I do truly miss coming home to hear you cursing at yourself for blowing up a capacitor or something.
The smell of burnt resistors is something I will never forget... In part because it hasn't truly gone away.
Leia has taken it over and has done everything she can to complete some of your old projects.
She read through your old copy of \textit{The Art of Electronics} like it was an elementary-level picture book.
they say every generation is smarter than the last, but she's on an entirely different level.

Most recently, she finished the analog drone you wanted us to build together.
Ah, actually, that was a pretty funny story.
I don't know if you remember, but your drone design... Well, it was the technological manifestation of the grim reaper.

So, picture this: 

It's mid-July.
Swealteringly hot.
Just broiling.
Like, so hot that the eggs were boiling while still in the fridge.

Leia bursts from the workshop with this ungodly contraption.
It's apparently straight from your old solidworks model, but boy-oh-boy did it look like it could kill someone.
It had a huge, foot square base where a breadboard was mounted on the top with several battery packs below.
Four long arms stretched from the center, holding somewhat large, dual-bladed fans.
There were, of course, no safety precautions.
No cage or anything protecting anyone from the fans.
If they started spinning, someone would probably die.

She then starts talking like you used to -- no less than 1000 words per second about how the stabilization circuit would have never worked in the original design or how she was happy she found the larger fans as otherwise there was no way to support the weight of the batteries while in the air.
I had to snap her out of her trance to ask if she had tested it out yet, to which she replied, ``No. I was hoping to do that now!''

I don't really know what emotions I felt in that moment, but I did \textit{not} want her to test it out.
I just \textit{knew} in my heart that something bad would happen, so I tried to gently persuade her against flying the drone by changing the conversation a bit.
I grabbed it from her and inspected it a bit before asking, ``So who came up with the design, you or Grampa?''

She bagan speaking while returning to the workshop, ``Well, Grampa did initially, but honestly none of his software worked at all, so I had to re-design certain parts of it.''
I could hear her rummaging for something in the other room as I took a closer look at the design.
Honestly, it was pretty light.
I could imagine it flying pretty well, all things considered.

Leia then returned with a very long, thin cord what looked like one of those ancient playstation controllers.
``What's that?''
I asked.

``Ah, the controller. It auto-stabilizes, but I want to be able to move it around...'' She then plugged the cord in to the breadboard on the top of the drone.

``Wait.'' I had to think about what was happening. ``You built a drone that needs a leash? Isn't the purpose of these things to take aereal photos?''
I thought for a second longer before realizing yet another fatal flaw, ``Also: isn't that cord a bit short? Won't the propellers hit you at that distance?''

``Nah, no way...'' She literally waved her hand at my complaint, ``It auto-stabilizes, so it won't move unless I tell it to. As long as I start at a safe distance and don't command it to literally attack me, everything will be fine.''
Her voice trailed off as she again started looking for something in the workshop.

``Right...'' I flipped the drone over in my hands and ran my fingers along the batteries.
They were tightly screwed into place.
In fact, the entire construction was top-notch.
Still, I was uncomfortable and pressed further, ``Ok, let's say it's windy. Can't it be blown right at you?''

``Nah, It auto-stabilizes...''

``Right, but what if you trip and fall, tugging it by the cord towards you?''

``The cord is short, but not that short. It's like 4 meters long...'' She then left the toolbox again, this time with a red toolbox. ``Besides, it will auto-stabilize. Again, it won't move anywhere unless \textit{I} tell it to.''

``Ok. Right. But what if it starts moving for some other reason, like...''

She cut me off, ``What part of auto-stabilization do you not understand?''

``Right...''

I was trying to figure out how to tell her how uncomfortable I felt with her flying the device, but in that moment, she snatched it from my hands, saying,
``Ok. I'm off to test it out!'' She said.

``Wait! Leia... I just don't know how...'' I trailed off as I saw her expression change a bit. She was so excited to fly the drone. I know you would have been excited too. I couldn't just tell you (both) not to do it.
So I sat down on a chair and sighed, saying, ``Look. This thing is dangerous. I don't want to be near anything that it could hit. Let's go out on the dock to test it out later today. Both of us.''

I haggled a bit and convinced her to try it out around sunset, but I was sweating bullets the entire afternoon for more reasons than just the heat,
I let her run off mid-afternoon to prepare everything, but told her not to fire it up until I was there with her.

As I made my way towards the dock later that day, I realized it truly was the perfect day for an experiment: a bit cloudy with calm water and -- most importantly -- no wind.
The heat had died down a bit, but I still brought a small cooler with some ice cream for after everything was over.
I remember the cicadas being particularly loud, really selling the summer ambiance.

Apparently, she had spent the entire time trying to find the most level part of the dock to launch from, while sweeping off any debris to make sure nothing could interfere with her experiment.
To be honest, she was incredibly precise.
She started by testing each motor individually before firing each one of in short bursts from at least 2 meters away.
Her head was flush to the ground as she determined the exact amount of thrust necessary for lift-off.

When she did so, she also went on a tangent about how it took way more power than she anticipated and she only had enough battery for maybe 30 minutes once the drone was airborne.
But after that, she looked at me to make sure I was ready for the next step -- an actual launch.

Now keep in mind, at no point in the process did she ask for \textit{permission} to launch this dangerous piece of ancient technology onto the world.
Her backwards glance was more of a playful guesture to make sure I was keeping up with her.
She just did it, and I watched.
I felt a little better after her thorough testing just moments before, but I was tense, ready to launch myself and grab her if anything went awry.

Luckily, it didn't.
The drone made an awfully loud whirring noise and then lifted itself about a foot off the ground.
She then tugged at it a bit with her leash and watched as it corrected itself a bit.
Actually, she did that a bunch, dancing around a bit as she did so.
She then looked at me, pointing to her drone and yelled, ``Auto stabilization! In an analog drone!''

To be honest, it's still out of my depth, but she was happy and I was happy for her.
I then cautiously walked behind her and asked how far it could go.

She said, ``Well, as far as the leash can go, I guess, so 4 meters?''

I told her to test it, so he pushed forward on the D-pad of her controller and we watched as it lurched forward a bit and the started moving at a slow pace.

It was then that we found out the fatal flaw in her leash plan.
She had failed to properly secure it to the breadboard, and with her tugging earlier, the leash was just barely attached at all.
As the drone flew farther away, it unplugged itself.
For a few seconds, neither of us noticed, but then I looked down and noticed the cord in the water while the drone was still slowly drifting away from us.

That might have been one of the scariest moments in recent memory.
If that thing hit \textit{anything}, literally anything at all, it could cause massive damage.
I didn't even want to think about what would happen if it hit any\textit{one}.

Luckily, it was travelling to sea and there were no boats out at this hour, so we just awkwardly watched as it drifted off into the sunset, slowly becoming quieter and quieter until we saw it fall into the water and stopped making noise entirely.
We didn't say a word until the satellites were out.

That's when Leia said, ``Well, I guess that was a success.''

We both laughed and sat on the beach while eating ice cream. Now I keep the controller right next to your old cell phone as one of my fondest memories.
We tried to find the drone, itself a bunch of different times... I even bought a new depth-finder with the hope of seeing it with the fish.
Leia tried to build a submarine, but got lost in the clouds a bit instead.

\sout{I thought about dusting off the old scuba gear to find it, but}\footnote{Just stop}

Honestly, a few days after you left, I found myself missing the smell of burnt resistors.
It's somehow sweet and bitter at the same time.
There's no way to describe it unless you've experienced it yourself, and the only way to experience it is by blowing something up.

Luckily, Leia has a knack for electronics and picked up right where you left off, albeit with about a decade gap.
