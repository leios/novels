\chapter{The Lecture}

There is a boundary between the known and unknown -- a bubble that encloses everything that humans have ever known.
Every thought ever thunk is safely inside this boundary, and the purpose of research is to gently nudge the zone of knowledge further.
It is essential to every major technological development and might be the most important thing any one person can do.

Of course, there are many different ways to do research.
Industrial labs, for example, are often made to turn a profit, and every study is meant to create a better product by some predetermined metric.
Academic research, on the other hand, is less structured and more free.
It is often the product of an individual's curiosity and driven by several like-minded people.

Many people claim that industrial research is \textit{applied}, while academic research is \textit{fundamental} or \textit{basic}.
That is to say that while industrial research is immediately important, academic research is important in a decade or so when engineers have caught up.

Both forms of research are essential to pushing humanity forward; however, only one is profitable in the short-term.
The other is often paid for by governmental spending.
Unfortunately, this means that fundamental research has become a political hot-topic where the public no longer supports funding general laboratories.


There is a bubble that encloses everything humans have ever known.
We, as humans, cannot survive outside of this boundary.

Research is the act of pushing the boundary of human knowledge forward one small step at a time.
It is fundamentally the most important task people are capable of, and it underlies every technological achievement humanity has produced.
More than that, anyone can do it.

All you need to do is pick up a research paper and start reading.
After a few months of going through a few papers a day, you should be able to see it: the boundary of human knowledge -- the bubble that encloses all that we have ever known.
After a few more months, you might even have an idea of your own, something no one has ever thought of before, something that can push that boundary forward.

It's hard work, filled with treacherous pitfalls at every step


The world is changing.
The world has always been changing and will always continue to change.
In many ways, the human experience is about adapting to and capitalizing on that change.
This is why research is so fundamental to humanity.
Someone, somehow needs to explore the unknown, pushing the boundary of human knowledge forward one step at a time.

Sure, there are bad papers.
There are scientists with poor work ethic.
There are studies with improper methods.
There are spurrious initiatives that don't lead anywhere,

But without out fundamental curiosity leading us to question everything, we would never have developed as much as we have.

As long as we strive to learn more and reach into the unknown, the world will contiue to change, and that's a good thing.
A really good thing.

\begin{noteblock}{If you couldn't tell, I am really trying to establish this character's narrative voice. Their opinions might not be your opinions, and that's ok!}
Anyone who stands in the way of progress, stands in the way of humanity, itself.
\end{noteblock}

