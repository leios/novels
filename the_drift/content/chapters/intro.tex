\chapter*{Introduction}

I'll be honest, I am not entirely sure how to write this, so I'll start with a disclaimer: this work is not meant to spark any political discussion.
It \textit{is} meant to discuss how we, as a society, engage with the research literature and is split up into 4 distinct strategies:

\begin{enumerate}
\item Rejection of science
\item Deification of researchers
\item Research with immediate gratification
\item Research without immediate gratification
\end{enumerate}

Each of these ideas will be discussed within the context of a greater, fictional universe.
I recognize that each of these topics can be stretched into theses in their own right, but my PhD is in computational (quantum) science, not sociology.
More than that, I recognize that the general public -- those I am trying to reach the most with this work -- do not read scientific literature.
For that reason, I chose to formulate my ideas not as a scientific hypothesis, but as an opinion stated in fiction.

I believe that statement requires re-emphasis: any ideas expressed within this work are purely my own opinions and ideas.
If you agree with them, great!
If you disagree, also great!
I want this work to spark discussion, but I don't want it to cause any sort of division.
As such, I will provide additional clarification where needed, such as academic citations or a some back-of-the-envelope calculations used to justify events in-universe\footnote{These statements will usually come in the form of footnotes, like these, but might also require separate figures or example}.
If you are interested in the narrative only, these statements can be safely ignored; however, as this work is primarily meant to show how we engage with research as a whole, I believe they are essential to understanding this work as a whole.

I recognize that no matter how hard I try, it will be impossible to ``future proof'' this work.
That is to say, if anyone is reading this in 2122 (100 years from the time of writing), there *might* be some addendums that are completely incorrect.
Honestly, I highly doubt anyone will be reading this novel at all, let alone people in the far distant future, so I'll focus on keeping things as accurate as possible with currently known science.
Also, keep in mind that this is a work of fiction.
I am not only trying to be scientifically accurate, but also point out the scientific innaccuracies that make this a work of fiction.

Finally, I will try to keep this novel available for free online\footnote{at https://github.com/leios/novels/the\_drift} under a Creative Commons License\footnote{In particular, a Creative Commons Attribution, Non-Commercial, Share-Alike (CC-BY-NC-SA) license. This means that anyone is free to use this work for anything they like, so long as it is not for a commercial purpose and any derivative works have the same license.
If people use this work, they must attribute me, Dr. James Schloss (Leios)}.
If possible, I would like to have discussion about this work available on github for everyone to see and take part in.
In fact, because of my licensing and openness, I am not going to try to publish this through traditional means; therefore, if you notice any typos, loose narrative threads, poor writing, etc, feel free to create an issue (or pull request online).
Again, I encourage any and all engagement with this work.
Also: because this work is freely available online, please do not pay for this unless you:
\begin{enumerate}
\item Want to support me, as a creator. Note that all proceeds from this work will go into funding my own independent research, creating new novels, and potentially creating a small research lab moving forward.
\item Want a physical (or ebook) copy
\item Are buying this for someone else as a gift\footnote{To be honest, I don't know why you would do that, but thanks for the support!}
\end{enumerate}

Ok, that's it for the foreword.
I know most people skip this part of the book entirely and instead opt to hop right into the content, but due to the nonconventional nature of this work, I thought it was appropriate to discuss everything ahead-of-time.

